%% This document created by Scientific Word (R) Version 3.5

%TCIDATA{LaTeXparent=0,0,PDEnotes.tex}


\chapter{Differential operators and coordinate systems}

\section{Coordinate systems}

Most commonly, Cartesian, cylindrical or spherical coordinates are used.

\begin{figure}[ptbh]
\centering\includegraphics[width=12cm]{figs/coord.eps}\caption{(a) Cylindrical
and (b) spherical coordinate system.}%
\label{fig:coord}%
\end{figure}

The coordinates can be converted between the different systems as shown in the
following table:

\bigskip%
\begin{tabular}
[c]{lllll}
&  & Cartesian & Cylindrical & Spherical\\
Cartesian & $x=$ &  & $\rho\cos\phi$ & $r\cos\phi\sin\theta$\\
& $y=$ &  & $\rho\sin\phi$ & $r\sin\phi\sin\theta$\\
& $z=$ &  & $z$ & $r\cos\theta$\\
Cylindrical & $\rho=$ & $\sqrt{x^{2}+y^{2}}$ &  & $r\sin\theta$\\
& $\phi=$ & $\mathrm{arctan}\left(  y/x\right)  $ &  & $\phi$\\
& $z=$ & $z$ &  & $r\cos\theta$\\
Spherical & $r=$ & $\sqrt{x^{2}+y^{2}+z^{2}}$ & $\sqrt{\rho^{2}+z^{2}}$ & \\
& $\phi=$ & $\mathrm{arctan}\left(  y/x\right)  $ & $\phi$ & \\
& $\theta=$ & $\mathrm{arccos}\left(  \frac{z}{\sqrt{x^{2}+y^{2}+z^{2}}%
}\right)  $ & $\mathrm{arctan}\left(  \rho/z\right)  $ &
\end{tabular}

\section{Differential operators}

\subsection{Nabla operator}

The Nabla operator is a vector differential operator, given in Cartesian
coordinates with the unit vectors $\mathbf{e}_{x}$, $\mathbf{e}_{y}\ $and
$\mathbf{e}_{z}$ by%
\[
\mathbf{\nabla}=\mathbf{e}_{x}\frac{\partial}{\partial x}+\mathbf{e}_{y}%
\frac{\partial}{\partial y}+\mathbf{e}_{z}\frac{\partial}{\partial z}.
\]

By coordinate transformation, we obtain in cylindrical coordinates
($x=\rho\cos\phi$, $y=\rho\sin\phi$, $z=z$)%
\[
{\mathbf{\nabla}=\mathbf{e}_{\rho}\frac{\partial}{\partial\rho}+\mathbf{e}%
_{\phi}\frac{1}{\rho}\frac{\partial}{\partial\phi}+\mathbf{e}_{z}%
\frac{\partial}{\partial z}}%
\]
and in spherical coordinates ($x=r\cos\phi\sin\theta$, $y=r\sin\phi\sin\theta
$, $z=r\cos\theta$)%
\[
{\mathbf{\nabla}=\mathbf{e}_{r}\frac{\partial}{\partial r}+\mathbf{e}_{\phi
}\frac{1}{r\sin\theta}\frac{\partial}{\partial\phi}+\mathbf{e}_{\theta}%
\frac{1}{r}\frac{\partial}{\partial\theta}.}%
\]

\subsection{Related differential operators}

The gradient is the Nabla operator applied to a scalar function $V(\mathbf{r}%
)$ of the three-dimensional space coordinate $\mathbf{r}$,%
\[
grad(V)=\mathbf{\nabla}V.
\]
The divergence and curl operators are acting on a vector function
$\mathbf{A}\left(  \mathbf{r}\right)  $. They can in Cartesian coordinates be
expressed as
\begin{align*}
div(\mathbf{A})  &  =\mathbf{\nabla A},\\
curl(\mathbf{A})=  &  \mathbf{\nabla}\times\mathbf{A}.
\end{align*}
The following identities hold:\newline
\begin{equation}
curl(grad\left(  V\right)  )=0, \label{eq:curlofgradient}%
\end{equation}%
\begin{equation}
div(curl(\mathbf{A}))=0. \label{eq:divergenceofcurl}%
\end{equation}
The \emph{scalar} Laplacian operator $\Delta$ is defined as
\begin{equation}
\Delta V=div(grad\left(  V\right)  ), \label{eq:divergenceofgradient}%
\end{equation}
which corresponds in Cartesian coordinates to $\Delta V=\mathbf{\nabla}^{2}V$.

The \emph{vector} Laplacian operator $\Delta$ is defined by the relation
\begin{equation}
curl\left(  curl(\mathbf{A})\right)  =grad(div(\mathbf{A}))-\Delta\mathbf{A}.
\label{eq:curlofcurl}%
\end{equation}

\subsection{\label{sec:coord}Coordinate transformation}

The operator expressions must give the same (physical) result, independent of
the coordinate system used. Note that besides the differential operators, also
the vector elements of $\mathbf{A}\ $have to be transformed into the
corresponding coordinate system.%

%TCIMACRO{\TeXButton{TeX field}{\begin{center}}}%
%BeginExpansion
\begin{center}%
%EndExpansion%

\begin{tabular}
[c]{llll}
& Cartesian & Cylindrical & Spherical\\
$grad(V)=$ & $\mathbf{e}_{x}\frac{\partial V}{\partial x}$ & {$\mathbf{e}%
_{\rho}\frac{\partial{V}}{\partial\rho}$} & ${\mathbf{e}_{r}\frac{\partial{V}%
}{\partial r}}$\\
& $+\mathbf{e}_{y}\frac{\partial V}{\partial y}$ & {$+\mathbf{e}_{\phi}%
\frac{1}{\rho}\frac{\partial V}{\partial\phi}$} & ${+\mathbf{e}_{\phi}\frac
{1}{r\sin\theta}\frac{\partial{V}}{\partial\phi}}$\\
& $+\mathbf{e}_{z}\frac{\partial V}{\partial z}$ & {$+\mathbf{e}_{z}%
\frac{\partial{V}}{\partial\theta}$} & ${+\mathbf{e}_{\theta}\frac{1}{r}%
\frac{\partial{V}}{\partial\theta}}$\\
$div(\mathbf{A})=$ & $\frac{\partial A_{x}}{\partial x}+\frac{\partial A_{y}%
}{\partial y}+\frac{\partial A_{z}}{\partial z}$ & $\frac{1}{\rho}%
\frac{\partial\left(  \rho A_{\rho}\right)  }{\partial\rho}+\frac{1}{\rho
}\frac{\partial A_{\phi}}{\partial\phi}$ & $\frac{1}{r^{2}}\frac
{\partial\left(  r^{2}A_{r}\right)  }{\partial r}+\frac{1}{r\sin\theta}%
\frac{\partial A_{\phi}}{\partial\phi}$\\
&  & $+\frac{\partial A_{z}}{\partial z}$ & $+\frac{1}{r\sin\theta}%
\frac{\partial\left(  A_{\theta}\sin\theta\right)  }{\partial\theta}$\\
$curl(\mathbf{A})=$ & $\mathbf{e}_{x}\left(  \frac{\partial A_{z}}{\partial
y}-\frac{\partial A_{y}}{\partial z}\right)  $ & {$\mathbf{e}_{\rho}$}$\left(
\frac{1}{\rho}{\frac{\partial{A_{z}}}{\partial\phi}-}\frac{\partial A_{\phi}%
}{\partial z}\right)  $ & ${\mathbf{e}_{r}}\frac{1}{r\sin\theta}\left[
\frac{\partial\left(  A_{\phi}\sin\theta\right)  }{\partial\theta}%
-\frac{\partial A_{\theta}}{\partial\phi}\right]  $\\
& $+\mathbf{e}_{y}\left(  \frac{\partial A_{x}}{\partial z}-\frac{\partial
A_{z}}{\partial x}\right)  $ & {$+\mathbf{e}_{\phi}\left(  \frac{\partial
A_{\rho}}{\partial z}-\frac{\partial A_{z}}{\partial\rho}\right)  $} &
${+\mathbf{e}_{\phi}}\frac{1}{r}\left[  \frac{\partial\left(  rA_{\theta
}\right)  }{\partial r}-\frac{\partial A_{r}}{\partial\theta}\right]  $\\
& $+\mathbf{e}_{z}\left(  \frac{\partial A_{y}}{\partial x}-\frac{\partial
A_{x}}{\partial y}\right)  $ & {$+\mathbf{e}_{z}\frac{1}{\rho}$}$\left[
{\frac{\partial{\left(  \rho A_{\phi}\right)  }}{\partial\rho}-\frac
{\partial{A_{\rho}}}{\partial\phi}}\right]  $ & ${+\mathbf{e}_{\theta}}%
\frac{1}{r}\left[  \frac{1}{\sin\theta}\frac{\partial A_{r}}{\partial\phi
}-\frac{\partial\left(  rA_{\phi}\right)  }{\partial r}\right]  $\\
$\Delta V=$ & $\frac{\partial^{2}V}{\partial x^{2}}+\frac{\partial^{2}%
V}{\partial y^{2}}+\frac{\partial^{2}V}{\partial z^{2}}$ & $\frac{1}{\rho
}\frac{\partial}{\partial\rho}\left(  \rho\frac{\partial V}{\partial\rho
}\right)  $ & {$\frac{1}{r^{2}}\frac{\partial}{\partial r}\left(  r^{2}%
\frac{\partial V}{\partial r}\right)  $}$+\frac{1}{r^{2}\sin^{2}\theta}%
\frac{\partial^{2}V}{\partial\phi^{2}}$\\
&  & $+\frac{1}{\rho^{2}}\frac{\partial^{2}V}{\partial\phi^{2}}+\frac
{\partial^{2}V}{\partial z^{2}}$ & $+\frac{1}{r^{2}\sin\theta}\frac
{\partial{\left(  \sin\theta\frac{\partial V}{\partial\theta}\right)  }%
}{\partial\theta}$\\
$\Delta\mathbf{A}=$ & $\mathbf{e}_{x}\Delta A_{x}$ & {$\mathbf{e}_{\rho
}\left(  \Delta A_{\rho}-\frac{A_{\rho}}{\rho^{2}}\right.  $} & ${\mathbf{e}%
_{r}}\left[  \Delta A_{r}-\frac{2A_{r}}{r^{2}}\right.  $\\
&  & {$\left.  -\frac{2}{\rho^{2}}\frac{\partial A_{\phi}}{\partial\phi
}\right)  $} & $\left.  -\frac{2}{r^{2}\sin\theta}\left(  {\frac
{\partial\left(  A_{\theta}\sin\theta\right)  }{\partial\theta}}%
+\frac{\partial A_{\phi}}{\partial\phi}\right)  \right]  $\\
& $+\mathbf{e}_{y}\Delta A_{y}$ & {$+\mathbf{e}_{\phi}\left(  \Delta A_{\phi
}-\frac{A_{\phi}}{\rho^{2}}\right.  $} & ${+\mathbf{e}_{\phi}\left(  \Delta
A_{\phi}-\frac{A_{\phi}}{r^{2}\sin^{2}\theta}\right.  }$\\
&  & {$\left.  +\frac{2}{\rho^{2}}\frac{\partial A_{\rho}}{\partial\phi
}\right)  $} & ${\left.  +\frac{2}{r^{2}\sin\theta}\frac{\partial A_{r}%
}{\partial\phi}+\frac{2\cot\theta}{r^{2}\sin\theta}\frac{\partial A_{\theta}%
}{\partial\phi}\right)  }$\\
& $+\mathbf{e}_{z}\Delta A_{z}$ & {$+\mathbf{e}_{z}\Delta A_{z}$} &
${+\mathbf{e}_{\theta}}\left(  \Delta A_{\theta}-\frac{A_{\theta}}{r^{2}%
\sin^{2}\theta}\right.  $\\
&  &  & $\left.  +\frac{2}{r^{2}}\frac{\partial A_{r}}{\partial\theta}%
-\frac{2\cot\theta}{r^{2}\sin\theta}\frac{\partial A_{\phi}}{\partial\phi
}\right)  $%
\end{tabular}%

%TCIMACRO{\TeXButton{TeX field}{\end{center}}}%
%BeginExpansion
\end{center}%
%EndExpansion

\section{Classification of partial differential equations}

The classification is analogous to ordinary differential equations. As an
example, we look at the Poisson equation%
\[
\Delta V=-\frac{\rho}{\epsilon}.
\]%

%TCIMACRO{\TeXButton{TeX field}{\begin{center}}}%
%BeginExpansion
\begin{center}%
%EndExpansion%

\begin{tabular}
[c]{l|ll}
& Definition & Poisson equation\\\hline
&  & \\
Order & Given by order of highest derivative & $2^{nd}$ order\\
& of the unknown function & \\
&  & \\
linear & unknown function and its derivatives & linear\\
& appear only linearly & \\
quasilinear & highest derivatives of unknown & \\
& function appear only linearly & \\
&  & \\
inhomogeneous & contains terms without unknown & inhomogeneous\\
& function or its derivatives & \\
homogeneous & contains no terms without & homogeneous for\\
& unknown function or its derivatives & $\frac{\rho}{\epsilon}=0$ (Lapace eq.)
\end{tabular}%

%TCIMACRO{\TeXButton{TeX field}{\end{center}}}%
%BeginExpansion
\end{center}%
%EndExpansion

\chapter{Important partial differential equations in electrical engineering}

\section{Electrodynamics}

\subsection{Electromagnetic fields}

\textbf{Maxwell's equations:} \newline Consider the following set of
equations:%
\begin{align}
curl(\mathbf{H})  &  =\mathbf{j}+\frac{\partial\mathbf{D}}{\partial t}%
\hspace{0.5cm}\text{ (Ampere's law),}\label{eq:amperelaw}\\
curl(\mathbf{E})  &  =-\frac{\partial\mathbf{B}}{\partial t}\hspace
{0.5cm}\text{(Faraday's law),}\label{eq:faradaylaw}\\
div(\mathbf{B})  &  =0,\label{eq:Maxwell3}\\
div(\mathbf{D})  &  =\rho. \label{eq:Maxwell4}%
\end{align}
with the constitutive relations (i.e., material equations), which are in a
simple model given by
\begin{alignat}{3}
\label{eq:constitutiveequations}\mathbf{B}  &  =\mu\mathbf{H}\hspace
{1cm}\mathbf{D}=\epsilon\mathbf{E}\hspace{1cm}\mathbf{j}=\sigma\mathbf{E}\\
&
\end{alignat}
$\mathbf{E},\mathbf{D},\mathbf{H}$ and $\mathbf{B}$ denote the electric field,
the electric displacement field, the magnetic field, and the magnetic
induction, respectively. Furthermore, $\mu,\epsilon,\mathbf{j}$ and $\rho$
denote the magnetic permeability, the electric permittivity, the current flux
density and the conductivity, respectively. The elegant set of equations, Eq.
(\ref{eq:amperelaw}), are the famous Maxwell's equations that govern -
together with the material equations, e.g. of the form Eq.
(\ref{eq:constitutiveequations}) - physical processes in electrical and
communications engineering, optics and many other areas of science.\newline 

\subsubsection{Electrostatics \newline }

Electrostatics is the sub-branch of electrodynamics where it is assumed that
the electric and magnetic fields do not change with time, i.e. $\partial
\mathbf{D}/\partial t=0$ and $\partial\mathbf{B}/\partial t=0$ , as well as
vanishing current density $\mathbf{j}=0$. It can be easily seen that in this
situation Maxwell's equations can be decoupled into independent electric field
and magnetic field equations. Assume all time derivatives vanish, i.e.
$\frac{\partial}{\partial t}=0$, then from Maxwell's curl equation (Faraday's
equation), Eq. (\ref{eq:faradaylaw}), it follows that $\mathbf{\nabla}%
\times\mathbf{E}=0$, which means that the electric field is irrotational and
from the identity, Eq. (\ref{eq:curlofgradient}), it follows that there exists
a scalar potential $V$ such that
\[
\mathbf{E}=-grad\left(  V\right)  ,
\]
where $V$ is called electrostatic potential function. Plugging this into the
divergence equation for the electric displacement field, Eq.
(\ref{eq:Maxwell4}), and assuming constant permittivity $\epsilon$, we obtain%

\begin{align*}
div\left(  \mathbf{D}\right)   &  =div\left(  \epsilon\mathbf{E}\right)
=\rho\\
&  \Rightarrow div\left(  \mathbf{E}\right)  =-\Delta{V}=\frac{\rho}{\epsilon
},
\end{align*}
which is is famously called \emph{Poisson's equation}. In the particular case
when there are no free charges, $\rho=0$, we obtain \emph{Laplace's equation},
i.e.,
\begin{align}
\Delta V(x,y,z) &  =-\frac{\rho}{\epsilon}\quad\quad\quad\text{''Poisson's
equation''} &  & \label{eq:poissonequation}\\
\Delta V(x,y,z) &  =0\quad\quad\quad\text{''Laplace's equation''} &  &
\label{eq:laplaceequation}%
\end{align}
\newline \begin{figure}[ptbh]
\centering\includegraphics[width=12cm]{figs/cond.jpg}\caption{Potential in a
cylindrical capacitor.}%
\label{fig:cond}%
\end{figure}

\subsubsection{Stationary flux field}

We again assume that $\frac{\partial}{\partial t}=0$, however allow for a time
independent current flux density, i.e. $\mathbf{j}(\mathbf{r})\neq0$. Let us
further take that $\mu=\text{const}$. We can satisfy Eq. (\ref{eq:Maxwell3}),
$div(B)=0$, for the magnetic field, $B$, if we assume that there exists a
vector field $A$, called the vector potential, which satisfies the relation:
\[
\mathbf{B}=curl\left(  \mathbf{A}\right)  .
\]
Plugging this into the Ampere's equation, Eq. (\ref{eq:amperelaw}), and using
the identity Eq. (\ref{eq:curlofcurl}), we can derive an equation for the
vector potential:
\[
\mathbf{curl}(curl\left(  \mathbf{A}\right)  )\equiv grad\left(
div(\mathbf{A})\right)  -\Delta{\mathbf{A}}=\mu\mathbf{j}.
\]
Here $\Delta$ is the vector Laplacian operator, which operates on a vector and
produces a vector. The important thing to notice here is that the magnetic
field $B$ is not uniquely determined from the vector potential $A$, since
$div\left(  {\mathbf{B}}\right)  =0$ will be satisfied for any $A^{\prime
}=A+\nabla\Xi(r,t)$ with $\Xi(r,t)$ an arbitrary scalar field, also known as a
gauge function. This additional degree of freedom in uniquely defining $A$
allows us to choose the scalar field $\Xi$ in such a way that the equations
are maximally simplified, also known as a gauge transformation. When we choose
$\Xi$ such that $div\left(  \mathbf{A}\right)  =0$ holds, it is said that the
electromagnetic field is in the Coulomb gauge. We can always transform an
arbitrary vector potential $A^{\prime}$ into the Coulomb gauge by using a
gauge function that satisfies $\Delta\Xi=div\left(  \mathbf{A}^{\prime
}\right)  $. Therefore, without loss of generality we can assume that
$div\left(  \mathbf{A}\right)  =0$ and hence
\[
\Delta\mathbf{A}=-\mu\mathbf{j}.
\]
As mentioned above the vector Laplacian transforms a vector into a vector and
in Cartesian coordinates it is simply a component-wise application of the
scalar Laplacian onto the components of $A$. However in different coordinate
systems this is not the case, see Section \ref{sec:coord}.\newline 

\subsubsection{Electromagnetic waves\newline }

Now, let us allow for the electromagnetic fields to vary in time and let us
assume that: $\epsilon=\text{const}$, $\mu=\text{const}$, $\sigma
=\text{const}$ and finally $\rho=0$. This corresponds to a linear medium with
no free charges. We can eliminate the magnetic field from Faraday's law, Eq.
(\ref{eq:faradaylaw}), by taking the curl of both sides of the equation and
substituting in Ampere's equation, Eq. (\ref{eq:amperelaw}), yielding
\[
curl(curl\left(  \mathbf{E}\right)  )=grad(div\left(  \mathbf{E}\right)
)-\Delta\mathbf{E}=-\mu\cdot curl\left(  \frac{\partial\mathbf{H}}{\partial
t}\right)  =-\mu\sigma\frac{\partial\mathbf{E}}{\partial t}-\mu\epsilon
\frac{\partial^{2}\mathbf{E}}{\partial t^{2}}.
\]
Furthermore, using the fact that for $\rho=0$ we have $div\left(
\mathbf{E}\right)  =0$, we obtain
\begin{equation}
\Delta\mathbf{E}=\mu\sigma\frac{\partial\mathbf{E}}{\partial t}+\mu
\epsilon\frac{\partial^{2}\mathbf{E}}{\partial t^{2}}.
\label{eq:electricwavebase}%
\end{equation}
Similarly, one can derive an analogous equation for the magnetic field
$\mathbf{H}$:
\begin{equation}
\Delta\mathbf{H}=\mu\sigma\frac{\partial\mathbf{H}}{\partial t}+\mu
\epsilon\frac{\partial^{2}\mathbf{H}}{\partial t^{2}}.
\label{eq:magneticwavebase}%
\end{equation}
\newline \begin{figure}[ptbh]
\centering\includegraphics[width=12cm]{figs/waveguide.jpg}%
\caption{Electromagnetic wave propagation in waveguide.}%
\label{fig:wave}%
\end{figure}\newline \textbf{Special cases}:\newline 

For an insulator $\sigma=0$ we simply obtain the classical wave equation
\[
\Delta\mathbf{E}=\mu\epsilon\frac{\partial^{2}\mathbf{E}}{\partial t^{2}},
\]
where $1/\sqrt{\mu\epsilon}=c$ is the velocity of light inside the material. \newline 

For a metalic conductor where $\sigma\gg\omega\epsilon$ holds, we can neglect
the second order derivative term to obtain
\begin{equation}
\Delta\mathbf{E}=\mu\sigma\frac{\partial\mathbf{E}}{\partial t}%
\end{equation}
with diffusion constant $D=1/\mu\sigma$, which corresponds in Cartesian
coordinates to a diffusion-type equation $\Delta E_{x,y,z}=D^{-1}%
\frac{\partial}{\partial t}E_{x,y,z}$ for each of the three field vector components.

\subsection{Homogeneous transmission line}

\begin{figure}[ptbh]
\centering\includegraphics[width=12cm]{figs/trans.eps}\caption{Equivalent
circuit of a short transmission line segment.}%
\label{fig:trans}%
\end{figure}

The equations for the electric current $i(x,t)$ and voltage $u(x,t)$ along a
homogeneous transmission line, e.g., a cable, can be derived from the
equivalent circuit of a short transmission line segment, as shown in Fig.
\ref{fig:trans}. We obtain the following differential equations:
\begin{align}
-\frac{\partial u}{\partial x}  &  =R^{\prime}i+L^{\prime}\frac{\partial
i}{\partial t},\\
-\frac{\partial i}{\partial x}  &  =G^{\prime}u+C^{\prime}\frac{\partial
u}{\partial t},
\end{align}
with $R^{\prime},L^{\prime},G^{\prime}$ and $C^{\prime}$ denoting the
resistance, inductance, conductance and capacitance of the transmition line
per unit length, respectively. Eliminating either $u$ or $i$ from the set of
coupled equations leads to the famous ``telegrapher's equation'',
\begin{equation}
\frac{\partial^{2}f}{\partial x^{2}}=L^{\prime}C^{\prime}\frac{\partial^{2}%
f}{\partial t^{2}}+(R^{\prime}C^{\prime}+L^{\prime}G^{\prime})\frac{\partial
f}{\partial t}+R^{\prime}G^{\prime}f, \label{eq:telegrapherequation}%
\end{equation}
where $f$ can denote either $i$ or $u$. Assuming that there are no losses,
i.e., $R^{\prime}=0$ and $G^{\prime}=0$, we arrive at%

\begin{equation}
\frac{\partial^{2}f}{\partial x^{2}}=L^{\prime}C^{\prime}\frac{\partial^{2}%
f}{\partial t^{2}},
\end{equation}
which is again a one-dimensional wave equation.

\section{Charge transport in semiconductors}

\subsection*{Drift-diffusion model}

The transport of carriers inside a semiconductors is most simply described in
the so called drift-diffusion model, where the electric current is the sum of
two different components, namely the drift current and the diffusion current.
The drift current arises in response to the electric field and the diffusion
current is simply a result of the redistribution of carriers from regions of
high concentration to regions of low concentration (i.e. diffusion) inside the semiconductor.

\subsubsection{Continuity equations\newline }

Charge conservation inside the semiconductor requires that the following
equations, also known as continuity equations, hold:
\begin{align}
\frac{\partial n}{\partial t}  &  =\frac{1}{e}div\left(  \mathbf{j}%
_{n}\right)  +g-r,\label{eq:continuity_n_charge}\\
\frac{\partial p}{\partial t}  &  =-\frac{1}{e}div\left(  \mathbf{j}%
_{p}\right)  +g-r. \label{eq:continuity_p_charge}%
\end{align}
In above equation system, $n$ and $p$ are the negative and positive charge
densities (in units $1/m^{3}$), $\mathbf{j}_{n/p}$ are the corresponding
current flux densities, $e$ is the elementary charge with $e\approx
1.602\times10^{-19}C$, and $g$ and $r$ the so-called generation and
recombination rates. Electrons and holes are for example generated thermally
or by optical absorption and recombine by radiative, Shockley-Read-Hall and
Auger recombination. The corresponding rates $g$ and $r$ are thus generally
functions of $n$ and $p$.

\subsubsection{Convective current densities\newline }

In the drift-diffusion model, the current densities can be written as:
\begin{align}
\mathbf{j}_{n}  &  =-e\mathbf{v}_{n}n+eD_{n}grad\left(  n\right)  ,\nonumber\\
\mathbf{j}_{p}  &  =e\mathbf{v}_{p}p-eD_{p}grad\left(  p\right)  .
\label{eq:jnp}%
\end{align}
Here $\mathbf{v}_{n/p}$ denote the velocity of negative/positive charges
inside the semiconducor, and $D_{n/p}$ the corresponding diffusion constants.
In general both $v_{n/p}$ and $D_{n/p}$ depend on the applied electric field,
but in the weak field limit it can be assumed that:
\[
\mathbf{v}_{n/p}=\mp\mu_{n/p}E\hspace{2cm}D_{n/p}=\mu_{n/p}\frac{k_{B}T}%
{e}\leftarrow\text{(Einstein's relations) }.
\]
Note that the parameter $\mu_{n/p}$ denotes the carrier mobility, i.e. how
easily do electrons or holes respond to an applied electric field ($\mu_{n/p}$
has nothing to do with the magnetic permeability $\mu$ mentioned in the
previous section).

\subsubsection{Total current conservation\newline }

The Poisson equation, Eq. (\ref{eq:poissonequation}), is in the semiconductor
given by%
\begin{equation}
div\left(  \mathbf{D}\right)  =\rho=e\left(  N_{\mathrm{D}}-N_{\mathrm{A}%
}+p-n\right)  , \label{eq:poissonequation2}%
\end{equation}
where $N_{\mathrm{D}}$ and $N_{\mathrm{A}}$\ are the densities of ionized
donators and acceptors, respectively. Taking the divergence of Eq.
(\ref{eq:amperelaw}),%
\[
div\left(  \mathbf{j}_{n}+\mathbf{j}_{p}+\frac{\partial\mathbf{D}}{\partial
t}\right)  =div\left(  rot(\mathbf{H})\right)  =0,
\]
and inserting the expressions for the current densities, Eq. (\ref{eq:jnp}),
yields with the Poisson equation, Eq. (\ref{eq:poissonequation2}),
\[
div\left(  \mathbf{j}_{n}+\mathbf{j}_{p}\right)  =-\frac{\partial\left[
div\left(  \mathbf{D}\right)  \right]  }{\partial t}=\frac{\partial\lbrack
e(n-p)]}{\partial t}.
\]
\newline \newline \emph{Majority carrier transport (e.g. electrons)} \newline 

In this regime of charge transport it is assumed that one type of carriers,
e.g. electrons, has a much higher density than carriers with the opposite
polarity. Let us assume for now that $g=r$ and $n\gg p$ which is a regime of
predominantly \emph{electron} transport through a semiconductor in equilibrium
(the equilibrium condition implies that $g=r$). Then the continuity equation
takes for constant $D_{n}$ the following form:
\[
\frac{\partial n}{\partial t}=\frac{1}{e}div\left(  \mathbf{j}_{n}\right)
=-div\left(  \mathbf{v}_{n}\right)  \cdot n-\mathbf{v}_{n}grad\left(
n\right)  +D_{n}\Delta n.
\]
For electric field modulation by an amplitude $\delta\mathbf{E}$, we insert
the ansatz%
\begin{align*}
\mathbf{E}\left(  \mathbf{r},t\right)   &  =\mathbf{E}_{0}+\delta
\mathbf{E}\left(  \mathbf{r},t\right)  ,\\
n\left(  \mathbf{r},t\right)   &  =N_{\mathrm{D}}+\delta n\left(
\mathbf{r},t\right)  ,\\
\mathbf{v}\left(  \mathbf{r},t\right)   &  =\mathbf{v}\left(  \mathbf{E}%
_{0}\right)  -\mu_{n}\cdot\delta\mathbf{E}\left(  \mathbf{r},t\right)
=\mathbf{v}_{0}+\delta\mathbf{v},
\end{align*}
yielding%
\[
\frac{\partial\left(  \delta n\right)  }{\partial t}=+\mu_{n}div\left(
\delta\mathbf{E}\right)  (N_{\mathrm{D}}+\delta n)-(\mathbf{v}_{0}%
+\delta\mathbf{v})grad\left(  \delta n\right)  +D_{n}\Delta\left(  \delta
n\right)
\]
In the weak field regime with $\delta n\ll N_{\mathrm{D}}$, $|\delta
\mathbf{v}|\ll|\mathbf{v}_{0}|$, we obtain with Eq. (\ref{eq:poissonequation2}).%

\[
\frac{\partial\left(  \delta n\right)  }{\partial t}=-\frac{e\mu
_{n}N_{\mathrm{D}}}{\epsilon}\delta n-\mathbf{v}_{0}grad\left(  \delta
n\right)  +D_{n}\Delta\left(  \delta n\right)  .
\]
The first term describes relaxation, the second term drift, and the third term
diffusion.\newline \newline \emph{Minority carrier transport}

For minority carriers, it can often be assumed that the current Eq.
(\ref{eq:jnp}) is dominated by the diffusion term. With Eq.
(\ref{eq:continuity_n_charge}), we then obtain for constant $D_{n}$ and $g=r$
\begin{equation}
\frac{\partial n}{\partial t}=D_{n}\cdot\Delta n, \label{eq:diff}%
\end{equation}
for electrons, and analogously for holes. This is a so-called diffusion equation.

\section{Thermodynamics}

In the theory of thermodynamics one of the fundamental equations, equivalent
to the energy/mass conservation equation is the continuity equation for heat:
\[
\frac{\partial w}{\partial t}=\rho c_{p}\frac{\partial T}{\partial
t}=-div\left(  \mathbf{j}_{w}\right)  +p,
\]
where $w$ denotes the heat energy density (in units energy per unit volume),
$\rho$ the mass density, $c_{p}$ the heat capacity, $\mathbf{j}_{w}$ the
energy flux density and $p$ the heat source density. For the energy flux
density due to heat conduction (i.e., neglecting heat transport by convection
and radiation) we can with the heat conductivity $\lambda$ write
\[
\mathbf{j}_{w}=-\lambda\cdot\mathbf{\nabla}T.
\]
Thus we have
\[
\rho c_{p}\frac{\partial T}{\partial t}=div(\lambda\cdot\mathbf{\nabla}T)+p.
\]
Without any heat sources ($p=0$), we obtain for $\lambda=$const%

\begin{equation}
\Delta T=\frac{\rho c_{p}}{\lambda}\frac{\partial T}{\partial t},
\end{equation}
which is the heat (conduction) equation or also Fourier's law of heat
conduction. This equation has the same form as the diffusion equation, Eq.
(\ref{eq:diff}).

\section{Schr\"odinger equation}

The microscopic description of electronic processes becomes more and more
important in electrical engineering, for example with respect to
nanostructured electronic devices. In the microscopic domain, the behaviour of
electrons is described by the so-called Schr\"{o}dinger equation, first
formulated in 1925 by the Austrian physicis Erwin Schr\"{o}dinger. Since it is
valid on a more fundamental level than classical physics, Schr\"{o}dinger's
equation cannot be strictly derived from it. However, a general schematic of
how one can obtain this fundamental law will be outlined in what follows.

From classical theory we know that the total energy of a particle with mass
$m$ is the sum of its kinetic and potential energy ($V(\mathbf{r})$):
\begin{equation}
E=\frac{\mathbf{p}^{2}}{2m}+V(\mathbf{r},t),\label{eq:hamiltonequation}%
\end{equation}
where $\mathbf{p}$ is just the momentum of the particle.

A year before Schr\"{o}dinger formulated his famous equation, the French
physicist Louis de Broglie postulated that matter may, under some special
conditions, exhibit wave-like behaviour (he based this conjecture on
previously observed experimental results). Furthermore, he held that the
wavelength of such a matter wave is related to its momentum by the following
relation (famously known as the de Broglie relation) : $\lambda=h/p$, where
$h=2\pi\hbar\approx6.626\,\mathrm{m}^{2}\mathrm{kg}/\mathrm{s}$ is the famous
Planck's constant and $\hbar$ is the reduced Planck's constant. In a more
general framework one can relate the vector momentum $\mathbf{p}$ to the wave
vector $\mathbf{k}$ of the matter wave, as well as its frequency $\nu$ to its
energy $E$ as:%

\begin{align}
\mathbf{p}  &  =\hbar\mathbf{k},\nonumber\\
E  &  =h\nu=2\pi\hbar\omega. \label{eq:debroglierelations}%
\end{align}

Now assuming that this matter wave can be written down as a simple plane wave
$\Psi(\mathbf{r},t)=\Psi_{0}\times\exp\left[  i(\mathbf{k}\cdot\mathbf{r}%
-\omega t)\right]  $ with amplitude $\Psi_{0}$, angular frequency $\omega$ and
angular wave vector $\mathbf{k}$ , one can easily verify the following
relations:%
\begin{equation}%
\begin{array}
[c]{cc}%
\frac{\partial\Psi}{\partial t}=-i\omega\cdot\Psi & \Rightarrow\omega
=i\frac{\partial}{\partial t},\\
\mathbf{\nabla}\Psi=i\mathbf{k}\cdot\Psi & \Rightarrow\mathbf{k}%
=-i\mathbf{\nabla}.
\end{array}
\label{eq:firstquantization}%
\end{equation}

Now using the above correspondences, Eq. (\ref{eq:firstquantization}), the de
Broglie relations, Eq. (\ref{eq:debroglierelations}), and the energy equation,
Eq. (\ref{eq:hamiltonequation}), we arrive at
\begin{equation}
i\hbar\frac{\partial\Psi\left(  \mathbf{r},t\right)  }{\partial t}%
=-\frac{\hbar}{2m}\Delta\Psi\left(  \mathbf{r},t\right)  +V\left(
\mathbf{r},t\right)  \Psi\left(  \mathbf{r},t\right)
.\label{eq:schroedingerequation}%
\end{equation}
Note that now the energy and the momentum of the particle are not mere numbers
but they are rather represented as operators, $E=i\hbar\frac{\partial
}{\partial t}$ and $\mathbf{p}=-i\hbar\mathbf{\nabla}$, which act on
$\Psi(\mathbf{r},t)$, the so called wave function, describing the elementary particle.

\begin{figure}[ptbh]
\centering\includegraphics[width=12cm]{figs/schrod.eps}\caption{Solution of
the time dependent Schr\"{o}dinger equation in a quantum well.}%
\label{fig:schrod}%
\end{figure}

\section{General properties}

The most important partial differential equations in electrical engineering
and physics are, as we have seen so far, of second order. In this lecture we
will thus treat mainly such types of equations, which can generally be written
in one of the following forms:%
\[%
\begin{array}
[c]{cc}%
\Delta u=-f(\mathbf{r}) & \leftarrow\text{ ''Potential'' equation,}\\
\Delta u=\frac{1}{c^{2}}\frac{\partial^{2}u}{\partial^{2}t} & \leftarrow\text{
''Wave'' equation,}\\
\Delta u=\frac{1}{D}\frac{\partial u}{\partial t} & \leftarrow\text{
''Diffusion'' equation.}%
\end{array}
\]

From the few examples of derivations and also from basic phyiscal
considerations it is clear that the potential equation will have meaningful
solutions only when correct boundary conditions are specified, whereas the
wave and diffusion equations require supplementary boundary and initial
conditions in order to be well-defined.